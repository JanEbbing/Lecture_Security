\documentclass[12pt]{article}
\usepackage{lingmacros}
\usepackage{tree-dvips}
\usepackage[utf8]{inputenc}
\begin{document}

\section*{L\"osungen f\"ur Sicherheit \"Ubungsblatt 2}
\subsection*{Aufgabe 1}
\subsubsection*{Fall 1: Der Initialisierungsvektor wird fest gewählt}
\emph{Anmerkung: Die Nachrichten sind prinzipiell beliebig}.\\
Wir konstruieren einen erfolgreichen Angreifer auf ein beliebiges Blockchiffre im CBC-Modus mit festem IV.
Unser Angreifer läuft folgendermaßen ab:
\begin{enumerate}
  \item Wir w\"ahlen als Nachrichten $M_0$, das nur aus Nullen besteht, sowie $M_1$, das nur aus Einsen besteht.
  \item Nutze das Orakel, um Enc(K, $M_0$) sowie Enc(K, $M_1$) zu berechnen.
  \item Verschl\"ussele eine zuf\"allige der beiden Nachrichten mit dem Verschl\"usselungsalgorithmus.
  \item Nach Konstruktion muss das Chiffrat, das der Angreifer erhält, genau eines der zwei sein, die er vorher berechnet hat. Gebe den dazu geh\"origen Klartext aus.
\end{enumerate}
Dieser Angreifer gewinnt das IND-CPA-Spiel immer.

\end{document}
