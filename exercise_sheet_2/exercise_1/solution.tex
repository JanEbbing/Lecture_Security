\documentclass[12pt]{article}
\usepackage{lingmacros}
\usepackage{tree-dvips}
\usepackage[utf8]{inputenc}
\begin{document}

\section*{L\"osungen f\"ur Sicherheit \"Ubungsblatt 2}
\subsection*{Aufgabe 1}
\subsubsection*{Fall 1: Der Initialisierungsvektor wird fest gewählt}
\emph{Anmerkung: Die Nachrichten sind prinzipiell beliebig}.\\
Wir konstruieren einen erfolgreichen Angreifer auf ein beliebiges Blockchiffre im CBC-Modus mit festem IV.
Unser Angreifer läuft folgendermaßen ab:
\begin{enumerate}
  \item Wir w\"ahlen als Nachrichten $M_0$, das nur aus Nullen besteht, sowie $M_1$, das nur aus Einsen besteht.
  \item Nutze das Orakel, um Enc(K, $M_0$) sowie Enc(K, $M_1$) zu berechnen.
  \item Verschl\"ussele eine zuf\"allige der beiden Nachrichten mit dem Verschl\"usselungsalgorithmus.
  \item Nach Konstruktion muss das Chiffrat, das der Angreifer erhält, genau eines der zwei sein, die er vorher berechnet hat. Gebe den dazu geh\"origen Klartext aus.
\end{enumerate}
Dieser Angreifer gewinnt das IND-CPA-Spiel immer.

\subsubsection*{Fall 2: IV wird fest gewählt und bei jeder Verschlüsselung um 1 hochgezählt.}
\emph{Anmerkung: \textasciitilde{}(W) bezeichne das bitweise Komplement eines Bitstrings W}\\
Wir konstruieren einen Angreifer auf ein beliebiges Blockchiffre im CBC-Modus mit konstanter IV-Wahl, der bei jedem Verschlüsselungsvorgang um 1 erhöht wird.
\begin{enumerate}
  \item
  \item
\end{enumerate}
\subsection*{Aufgabe 2}
\subsection*{Aufgabe 3}
\subsection*{Aufgabe 4}
\begin{enumerate}
\item WPA 2 (benutzt AES): \begin{itemize}\item Blockchiffre \item Counter Mode with CBC-MAC (CTR) \item IV im Klartext in der Nachricht: IV = Priorität (immer 0, noch), Padding, MAC-Source-Address, Package Number \item Schlüssel: Pre-shared key, den jede Station von vornherein kennen muss. Daraus werden temporäre Schlüssel berechnet die dann zur Verschlüsselung benutzt werden. \end{itemize}

\end{enumerate}

\end{document}
